% !TEX TS-program = xelatex
% !TEX encoding = UTF-8 Unicode
% !Mode:: "TeX:UTF-8"

\documentclass{resume}
\usepackage{zh_CN-Adobefonts_external} % Simplified Chinese Support using external fonts (./fonts/zh_CN-Adobe/)
%\usepackage{zh_CN-Adobefonts_internal} % Simplified Chinese Support using system fonts
\usepackage{linespacing_fix} % disable extra space before next section
\usepackage{cite}

\begin{document}
\pagenumbering{gobble} % suppress displaying page number

\name{唐亮}

\basicInfo{
  \email{shinytang6@gmail.com} \textperiodcentered\ 
  \phone{(+86) 18221865643} \textperiodcentered\ 
}
 
\section{\faGraduationCap\  教育背景}
\datedsubsection{\textbf{上海交通大学}, 上海}{2015.9 -- 至今}
\textit{学士}\ 信息安全, 预计 2019 年 6 月毕业

\section{\faBuilding\ 实习经历}
\datedsubsection{\textbf{旷视科技 Face++} 北京}{2018.7 -- 至今}
\role{暑期实习}{研发实习生}
\begin{itemize}
  \item 在旷视研究院人脸识别组任职, 从事聚类方面的工作
  \item 探索图片聚类以及优化代码
\end{itemize}

\datedsubsection{\textbf{谷歌编程之夏 Google Summer of Code} 远程}{2018.5 -- 至今}
\role{夏令营}{开源贡献者}
\begin{itemize}
  \item 本次GSoC大陆地区共有52人入选
  \item 维护Processing基金会旗下的p5.js网页编辑器,为其开发用户友好的控制台
\end{itemize}

\datedsubsection{\textbf{零号湾} 上海}{2017.6 -- 2017.8}
\role{暑期实习}{软件开发实习生}
\begin{itemize}
  \item 实现了一个高精细度权限控制系统
  \item 维护/开发零号湾官网以及公司官网
\end{itemize}

\section{\faUsers\ 项目经历}
\datedsubsection{\textbf{Cluster}}{2018.7 -- 至今}
\role{Python}{}
\begin{itemize}
  \item 探索图片聚类算法,优化代码
\end{itemize}

\datedsubsection{\textbf{p5.js web editor}}{2018.2 -- 至今}
\role{Mongodb, Express, React/Redux, Nodejs}{开源贡献者}
\begin{itemize}
  \item 维护/开发开源项目p5.js网页编辑器
\end{itemize}
项目链接: https://github.com/processing/p5.js-web-editor

\datedsubsection{\textbf{Sniffer}}{2017.9 -- 2018.1}
\role{C++}{个人项目}
\begin{itemize}
  \item 实施了一个带有界面的网络抓包嗅探器 (Qt5, Wincap)
  \item 支持包过滤/查询, 导入/导出数据包, 重组TCP包
\end{itemize}
项目链接: https://github.com/shinytang6/Sniffer


\datedsubsection{\textbf{Campus-Information-Platform}}{2017.11 -- 2017.12}
\role{Laravel, Vue}{小组项目}
该项目的初衷是帮助在校大学生更好地收集校内外的活动信息
\begin{itemize}
  \item 发布校内外的大小活动信息
  \item 支持订阅等功能
\end{itemize}
项目链接: https://github.com/SE407-2017/campus-information-platform

\datedsubsection{\textbf{Logico}}{2017.11}
\role{Golang}{个人项目}
该项目的初衷是为了优化阅读体验, 创建于 Go Hack 2017
\begin{itemize}
  \item 根据文章中不同词的词性渲染出不同的颜色
\end{itemize}
项目链接: https://github.com/shinytang6/Logico


\datedsubsection{\textbf{Insigter}}{2017.10}
\role{Python, Vue}{个人项目}
健康评估系统, 创建于 hackxfdu
\begin{itemize}
  \item 基于用户在社交媒体上发的动态来分析其心理健康程度,并给出一个直观的健康指数
\end{itemize}
项目链接: https://github.com/hack-fdu/Insighter

\datedsubsection{\textbf{Tango}}{2017.6 - 2017.8}
\role{Golang, Vue}{个人项目}
\begin{itemize}
  \item 一个高精细度权限控制系统
  \item 其目的是用作企业内部文件管理系统
\end{itemize}

\datedsubsection{\textbf{imoocMovie}}{2017.5}
\role{Nodejs}{个人项目}
慕课网电影网站\\
项目链接: https://github.com/shinytang6/imoocMovie

\datedsubsection{\textbf{studentManageSystem}}{2017.4 -- 2017.5}
\role{PHP}{小组项目}
项目链接: https://github.com/shinytang6/studentManageSystem

\\
\\
更多项目链接: https://github.com/shinytang6


% Reference Test
%\datedsubsection{\textbf{Paper Title\cite{zaharia2012resilient}}}{May. 2015}
%An xxx optimized for xxx\cite{verma2015large}
%\begin{itemize}
%  \item main contribution
%\end{itemize}

\section{\faCogs\ IT 技能}
% increase linespacing [parsep=0.5ex]
\begin{itemize}[parsep=0.5ex]
  \item 编程语言: JavaScript (ES6, Vue/Vuex, React/Redux, Nodejs) > Golang (Beego) > Python > PHP > C/C++ > Verilog > Java
  \item 数据库: Mysql > Mongodb
  \item 版本控制: Git
  \item 平台: Windows/Linux
  \item 开发: Web (全栈)
  \item 其他: 熟悉 TCP/IP 和网络编程
\end{itemize}

\section{\faHeartO\ 获奖情况}
\datedline{\textit{第三名}, i-lab创客马拉松}{2017.12}
\datedline{\textit{三等奖}, 全国大学生数学建模大赛}{2017.9}
\datedline{\textit{第五名}, hack.init()创客马拉松}{2017.7}
\datedline{\textit{第三名}, 英特尔物联网大赛}{2016.10}
\datedline{\textit{二等奖}, 上海交通大学奖学金}{2016.9}
\datedline{\textit{三等奖}, 全国高中数学联赛}{2014.10}
\datedline{\textit{一等奖}, 上海市TI数学竞赛}{2013.6}


\section{\faInfo\ 其他}
% increase linespacing [parsep=0.5ex]
\begin{itemize}[parsep=0.5ex]
  \item GitHub: https://github.com/shinytang6
  \item 语言: 英语 - 熟练
  \item 自我评价:热爱/拥抱开源的技术爱好者
\end{itemize}

%% Reference
%\newpage
%\bibliographystyle{IEEETran}
%\bibliography{mycite}
\end{document}
